PV-microgrids are proposed as a solution for communities with good solar conditions, but low access to energy infrastructure. A provider of these, Differ Community Power (DCP), wants to introduce an energy management system to their microgrids to increase the life-time and satisfaction from their systems. This thesis looks at an existing system supporting a medical facility in rural Malawi, analyses its operations and proposes an energy management system. The system is tuned to achieve goals developed in collaboration with DCP and the end-user. These goals include a higher availability of critical loads and better battery health. The proposed solution is compared to the existing system in a simulated environment, showing an increase in critical load reliability of up to 2.86 percentage points and a reduction in time in states damaging for the battery of up to 73\% dependent on the tuning. The proposed control system is also tested against the current control system under decreasing amount of battery capacity installed. By a 53\% reduction of battery capacity, the proposed control has a higher critical load reliability of 7.1 percentage points compared to the current system.
\newline
\newline
\newline
PV drevne mikrogrids er en mulig løsning for elektrifisering av områder med gode solforhold der kobling til annen energi-infrastruktur er kostbart eller krevende. Differ Community Power (DCP) er en utvikler av slike mikrogridanlegg. De ønsker å ta i bruk kontrollsystemer for å bedre tilfredsheten og livstidene til systemene deres. Denne oppgaven ser på et av disse anleggene i rurale Malawi og, etter å ha analysert det, foreslår et nytt kontrollsystem. System er stilt til å tilfredstille kriterier utvilket sammen med DCP og deres brukere. Dette inkluderer mål på sikkerheten til kritisk last og batteri-levetid. Det foreslåtte systemet sammenliknes med det eksiterende systemet i et simulert miljø, der det foreslåtte systemet viser en forbedring på tilfredsstillende av kritisk last på 2.86 prosentpoeng og en reduksjon i tiden batteriet er i skadelige tilstander på 73\% avhengig av parametersettingen til systemet. Det foreslåtte kontrollsystemet er også testet mot det eksisterende kontrollsystemet i simulering der installert batterikapasitet er redusert. Med en 53\% reduksjon i batterikapasitet, så viser det foreslåtte kontrollsystemet en forbedring i tilfredstillese av kritisk last på 7.1 prosentpoeng sammanliknet med eksisterende kontrollsystem. 




