The topic of microgrid control encompasses several related fields of research such as energy control systems, load and production forecasting and system sizing. On the specific topic of energy control systems for microgrids, studies vary in methodology, control level and in the systems considered. \textit{(Abhishek A. et al. 2020)} outlines in a review paper how microgrid control is usually separated into a 3-layered functionality-based hierarchy. The two initial levels, primary and secondary, work within the microgrid itself to maintain voltage and power balances. The highest level, tertiary, is used to optimize the operation of the grid based on cost, utilization, prioritization etc. The term \textit{Energy Management System} (EMS) is often used interchangeably for the tertiary level. The implementation of the hierarchical control into hardware can be both centralised and decentralised. This thesis will consider tertiary control without concern for hardware implementation.\cite{Abhishek2020-ox}\\

The 2023 review paper by \textit{(Allwyn R.G. et al. 2023)} explores different approaches to tertiary control based on the availability of grid connection and the number of different energy sources connected. If multiple energy sources are connected, such as a system consisting of both solar, wind, battery and grid connection, the operation cost and characteristics of each source are important to consider to achieve optimal operation. Although the site considered in this thesis is only PV-powered, an optimal operation of several energy sources is still relevant for the system considered in this thesis due to the operating cost of the battery. The paper also explores the control from the demand side, known as \textit{demand side management} (DSM).\cite{Allwyn2023-pd}\\

DSM is the modification of consumption to improve the operation of the microgrid. In cases where the production sources are inflexible, such as the case in this thesis, DSM is the only available method to shape operation. It is often performed by implementing price mechanisms to incentivize consumers to shift their consumption to periods more suitable for the microgrid operation\cite{Kanakadhurga2022-op} or by introducing more energy efficient loads\cite{Gilda2018-nh}. \textit{(Wang et al., 2021)} however, argues against the use of price control for DSM in impoverished communities because of the lack of economic flexibility of the consumers. They propose a classification algorithm to classify consumption based on severity and use targeted approaches to guide consumers into more desirable behaviour. They do not, however, consider an automation of this guidance into a control system that could adjust load depending on the system conditions.\cite{Wang2021-kb}\\

Of the DSM-methods found in the literature survey, most use  a price mechanism to influence behaviour, or the replacement of current loads with more energy-efficient ones. Neither of these are relevant to this thesis which seeks to perform active demand side management by controlling existing loads. The papers by \textit{(Rajbhandari et al., 2022)} and \textit{(Philipo et al., 2022)} however, have a similar case as in this thesis. \textit{(Rajbhandari et al., 2022)} considers a rural microgrid in Nepal and conducts a user survey to classify loads into three levels of priority for the user. A function for user satisfaction based upon load allocation is then used both for allocating energy the day ahead and in real-time. Their control system uses an exhaustive search method to scan through all possible combinations of load and the resulting value of the user satisfaction function. A rule-based method is used to shed low-priority loads if their demand conflicts with the ability to provide for higher-priority loads later and hence decrease user satisfaction. In their experience, the total amount of loads served is reduced, although the system can serve more high-priority loads, leading to a higher user satisfaction score on their user satisfaction function.  The proposed control system solution illustrates an effective and intuitive method to both  control and analyse a system from the demand side. However, there are issues with their design. An extensive search method across all load combinations across the whole time-span will have a large and rapidly increasing complexity. Furthermore, while their solution classifies loads based on priority, other attributes, such as the ability to shift demand are not considered. Loads, where the demand is flexible offer more control options. Neglecting this, as done in their solution, might provide a sub-optimal solution.\cite{Rajbhandari2022-oo}\\

On the other hand, \textit{(Philipo et al., 2022)}  does include both the priority and other key control characteristics in their load analysis of microgrids in East Africa. The two key characteristics were defined to be the ability to shift demand in time, \textit{(deferability)}, and interrupt a load after starting it, \textit{(interuptability)}. Their goal was to perform load shifting and peak clipping to have the load curve fit better the PV-production curve. This they achieved through an artificial neural network algorithm that used the irradiance and expected demand curves as input and produced a real-time updated scheduling of loads. The result was a decrease in peak-demand and peak-to-average ratio (PAR) of 31.2\% and 7.5\% respectively. The paper highlights the possible gains by classifying and controlling loads based on more characteristics than just their priority. However, their solution did not deal with a larger set of control objectives such as battery lifetime.\cite{Philipo2022-rx}\\

When several energy sources are present, the problem of controlling these is known in the literature as energy management. Common for microgrids, an EMS is tasked with optimally combining a renewable and intermittent energy resource, such as wind or solar, with a dispatchable resource such as a diesel generator or grid. Both \textit{(Sadek SM. et al., 2020)} and \textit{(Salazar A. et al., 2020)} consider a microgrid system of this type in their papers, and aim to construct an EMS minimizing fuel cost. In \textit{(Sadek SM. et al., 2020)} a microgrid system consisting of both a wind and PV-module together with a diesel generator is modelled based on both their active and reactive power. A non-linear optimization problem is constructed aiming to minimize fuel cost, the cost of shedding loads and the cost of curtailing renewable resources. They find an improvement in results when considering reactive power as opposed to when not. While the non-linear optimisation problem set-up is relevant, the scale and inclusion of reactive power is out of the scope of this thesis.\cite{Sadek2020-wl}

The paper by \textit{(Salazar A. et al., 2020)} has a smaller system and range of objectives. They have also included a methodology for production forecasting. In their study, they developed and compared the performance of a rule-based control method with a non-linear optimization approach on a system with photovoltaic power, batteries and a fuel-based generator to support a residential load. The control system combines an optimal energy management problem with a stochastic formulation of the PV-production. The optimal energy problem is formulated as a \textit{receding time horizon} (RTH) non-linear optimization problem where the power to and from the battery is the selection variable and fuel-based power generation is minimized. The dynamic nature of PV-production is captured by a Markov model. The optimization problem is solved by dynamic programming. Compared to the rule-based control, the stochastic optimal energy management system managed to decrease both generator usage and increase battery power availability.\cite{Salazar2020-al}
Both of these studies provide a relevant control method with their use of non-linear optimization. The cases differ though from the one in this thesis, as both papers include multiple sources of production.\\  

Quite a few studies have looked at various methods for load forecasting in microgrids. \textit{(Dutta et al., 2017)} Used a simple persistence technique to forecast both load and production in a microgrid. Their forecasts used the average of several days prior as look-back time. In their experiment, they vary both the look-back time and the forecast horizon. Their results showed a Mean Absolute Percentage Error of $2.42\%$ for the load forecasting with a look-back time of one time-step. Their method was poor in responding to changing conditions, especially for weather-dependent power prediction.\cite{Dutta2017-oi}\\

In a more complicated approach, \textit{(Zuleta-Elles et al, 2021)} used and compared a \textit{Auto Regressive Integrated Moving Average} (ARIMA) model and a \textit{Artificial Neural Network} (ANN) model to forecast demand in a microgrid. They developed different ARIMA models for the various forecasting horizons, some also with a seasonal component, turning it into a seasonal ARIMA (SARIMA) model. The ANN model developed was a 3-layer model with 96 regressors, each representing the load consumption within the past 24 hours, meaning 96 blocks of 15 minutes. Over various forecast horizons, they compared the best ARIMA model to the ANN model in terms of RMSE and MAE. Their results showed that the ANN model outperformed the ARIMA on all prediction horizons except 12 hours ahead. \cite{Zuleta-Elles2021-qh}\\

There are several approaches to PV-production forecasting, including statistical, physical and machine learning. \textit{(Huang et al., 2021)} did a comparative study between a physical model and a neural net model that they developed. The physical consisted of an irradiance model based on the position of the sun, and the solar panels modelled as a diode. They tested their model under both numerical weather predictions and with measured irradiance and temperature. When comparing the physical model to the neural net model, the physical model performed worse when receiving just numerical weather predictions, but better when receiving irradiance- and temperature measurements.\cite{Huang2010-lv} Their method did however not consider system losses, meaning energy lost within the system, which is a key factor in reducing available production for PV-microgrids. 

In the review paper from \textit{(Maghami et al., 2016)} the authors identify several loss factors including losses from shading, wiring and soiling. Of these, the authors find that shading has the potential for the largest losses, ranging between 10-70\% in some studies. The problem of soiling is found to be dependent on the region, where some regions, including sub-Saharan Africa, experience a high degree of soiling due to high dust intensity. The effect of soiling on the production depends on the angle and thickness of the dust layer, ranging from 1-26\%.\cite{Maghami2016-pq} 

In the 2011 paper by \textit{(Chimtavee A. and Ketjoy, N, 2012)} the authors perform a case study on a PV-microgrid system in Thailand. Over a year, they measured the irradiance and the energy produced by the PV-system. Their results showed an average loss of $26.27\%$ comparing the expected energy given measured irradiance to the actual energy produced.\cite{Chimtavee2012-gg}\\ 



From the literature study, the necessity of classifying loads both into their importance and control characteristics, as done in \textit{(Philipo et al., 2022)}\cite{Philipo2022-rx}, is considered valuable input for the control system designed in this thesis. Similarly, it builds upon the usage of a SARIMA-model as in \textit{(Zuleta-Elles et al, 2021)}\cite{Zuleta-Elles2021-qh} for load classification and the physical forecasting described by \textit{(Huang et al., 2021)}\cite{Huang2010-lv}. The physical model is modified based on the loss findings by \textit{(Chimtavee A. and Ketjoy, N, 2012)}\cite{Chimtavee2012-gg}. The receding time horizon optimization from \textit{(Salazar A. et al., 2020)}\cite{Salazar2020-al} is taken as inspiration for the optimizer in this thesis, although the  control objectives and options in their case differ from those in this thesis.\\

The literature survey yielded no study on PV-microgrids without generators or grid connection which combines load shaping and prioritization and battery charge management. This thesis builds upon insight into load and production analysis and forecasting, and energy management systems, and combines this into a, to the best of the author's knowledge, novel application.




