When I enrolled into a 5 year master degree in Engineering Cybernetics at NTNU in the Fall of 2018 I came with a strong interest in robotics and industrial automation. I even choose my specialization within Robot Systems. So why is that I, 5 years later write my thesis on Control of Energy Systems? I attribute this to two key events, the first of which being the Russian Invasion of Ukraine in February 2021 and the resulting shock to our Energy Systems. This event highlighted to me how energy is not to be taken for granted. It made me aware of another consideration to the Energy Transition: Energy Security. Energy Security means the ability to access energy and the reliability and robustness of these sources. The matter of Energy Security differs a lot between regions of the world.  Which brings me to my the second event that steered my academic trajectory into writing this thesis.\\

In waiting to go on an exchange, I was fortunate to be employed at DCP. This employment became prolonged, due to the COVID19-restrictions of my about-to-be host country. The silver lining, I got to travel to Malawi and see the impact of DCPs work. In rural Malawi, energy security and access to energy is no theoretical concern. It is a deep and enduring problem felt every day. When energy is delivered to communities which previously have not had access, the appreciation is deep and profound. 
At the health facilities, nurses told about how in the past cesarean sections at night had to be done with a cell phone light or candlestick in hand. The impact energy can have is literally a matter of life and death.\\

With these experiences in hand, I am very grateful to DCP for allowing me to write my master degree on a topic close to my heart, and that can have real world impact on underserved communities. I want to thank all DCP employees that have through discussions and feedback helped me shape and express my ideas. 

Last,but not least, I want to express my gratitude to my thesis advisor, Geir Mathisen for diligently guiding me throughout the process. 
