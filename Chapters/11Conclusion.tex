This thesis was tasked with the specific goal of developing a control system yielding an improved operation in some important aspects of DCPs microgrids. Through dialogue with both DCP and the end-users at site, a set of Key Performance Indicators was developed, enabling the evaluation of a proposed control system compared to the current one. These include an increased reliability for loads critical to the function of the site and the improved health of the battery installed in the microgrid. A proposed solution was developed based on analytics of the current operation at the specific site of Chiwoza. Simulating this solution over novel historical data yielded improved results on some of the pre-defined quantitative metrics, relating to battery life-time and critical load satisfaction, compared to the current control system. However, the proposed control system performed worse at other KPIs such as the utilization and total load reliability. The solution also satisfied a set of qualitative specifications, such as the adaptability to changing conditions and changing prioritization of objectives, the utilization of historical data and the ability to operate at a lower time frequency than the electrical system. The specification of applicability across various sites was partially met, although not confirmed through the experiment\todo{Maybe rephrase}.\\

The proposed solution was developed based on an analysis of the present operation. This included a statistical study of the consumption, which yielded insight into different consumption patterns for different loads, and a classification of loads based on their control characteristics. A user survey was developed and conducted to gain more insight into consumption, specifically about how the end-users prioritize amongst loads. Due to the weaknesses of the historical production data, the production was instead analysed from a physical perspective. Together, these two analytical studies yielded a demand and production forecaster, using a statistical and physical model respectively. A non-linear optimization controller was developed which, based on the forecasts and measurements of battery conditions, developed and implemented a plan for load allocation and battery charge/discharge. This control system was implemented and simulated in Matlab\cite{MATLAB}. \\

The work provides DCP with a methodology to control their operation for specific goals. It allows the company to gain insight into their consumption and production, and forecast that into the future. It can reduce their cost of equipment and maintenance, while providing them with more of a basis to guarantee certain performance levels to potential customers. For the end-users at the site, the work can, if implemented, allow them to have more predictability and assurance for the operation of loads critical to the facility.\\

In a larger sense, the work performed in this thesis is aligned with the UNs Sustainable Development Goal 7\cite{un_sdg_7}, which opened this thesis. The proposed control system provides a pathway towards more reliable and affordable renewable energy for a specific site, which constitutes a small portion of the \\
\newline
\newline
\newline
"\textbf{ENSURE ACCESS TO AFFORDABLE. RELIABLE.
SUSTAINABLE AND MODERN ENERGY FOR ALL}" \textit{(UN DESA, 2023, p.13-14)}\cite{Desa2023-mr} 