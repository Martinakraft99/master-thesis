The evaluation criteria outlined in table \ref{tab:KPI} form the baseline of the quantitative evaluation of the study. Looking at the results in table \ref{tab:combined_results_load}, an increase of 2.86  and 2.66 percentage points for the critical load reliability is found between the current control system and the proposed control system with tuning 1 and tuning 2 respectively. Relating this to KPI $V$ in table \ref{tab:KPI} this indicates an improvement in the ability of the proposed control system compared to the current control system to support loads critical to the functioning of the facility. The critical load reliability from the proposed control system is also better than the current when decreasing battery capacity, as seen in figure \ref{fig:cr_reliability}. In the figure, the reduction in critical load reliability of the proposed control system is only 1.6 percentage points when going from 7.5kWh to 3.5kWh battery capacity. This is superior to the current control system, in which critical load reliability drops by 7 percentage points over the same decrease in battery capacity.  

Given that the user survey in section \ref{seq:user_survey} found that purpose loads were given the highest priority of the end-user and that most users would prefer securing critical loads at the expense of non-critical loads, the increased critical load reliability indicates a potential increase in KPI $III$ - \textit{User Satisfaction}. The picture here is less clear however due to the decrease in total reliability of a little more than 1 percentage point for the proposed control system compared to the current control system. This decrease in total reliability is also evident in the reduction in utilization of the proposed control system compared to the current control system. With regards to the KPI $IV$, Utilization, the proposed control system is therefore inferior.\\

Using KPI $I$ and $II$ to evaluate the results with regards to battery health, table \ref{tab:combined_results_battery} shows that both the proposed control system has a reduction in both the amount of time the battery is above 90$\%$ of its capacity and when the charge/discharge rate exceeds the optimal threshold of 0.2C. The proposed control system is therefore successful in achieving better battery health than the current control system when measured on the KPIs in table \ref{tab:KPI}. The results for the proposed control system with tuning 2, shown in figure \ref{fig:bANDpb02_231224-240101_tuning2_soiling26}, have a larger reduction than the proposed control system with tuning 1, shown in figure \ref{fig:bANDpb02_231224-240101_tuning1_soiling26}. As tuning 2 was tuned to achieve better battery health than tuning 1, the lower values show that the tuning did achieve that.\\

The degree to which the proposed control system fulfils its specifications, found in table \ref{tab:control_system_specification}, is to be evaluated qualitatively. Two of these, numbers $1$ and $2$ are fulfilled by design. The proposed control system uses historical data and runs at a slower frequency than the sampling of demand and production. The improvement along some of the KPIs indicates that the proposed control system operates successfully while satisfying these two specifications.

Another of the qualitative evaluation criteria of the control system is number $3$, the ability to adjust to a change in the prioritization of goals. Tuning 1 was tuned to achieve high critical load reliability. Comparing tuning 1 and 2 in table \ref{tab:combined_results_load} shows that tuning 1 achieved higher critical load reliability than tuning 2. On the other hand, tuning 2 which was tuned for better battery health, did achieve better battery health results in table \ref{tab:combined_results_battery}. Because the two tunings did manage to achieve better performance than the other on the specific goal it was tuned for, this indicates that the proposed control system is adjustable to a change in goal prioritization. \\

Another qualitative criterion is the adaptability to changing conditions, shown as number $4$ in table \ref{tab:control_system_specification}. The current control system is by design not adaptive to changing conditions because the only behavior change is time-controlled. Forecasts do not feature as an input, leading to some of the weaknesses discussed in section \ref{seq:current_syst}, such as the inability to fully charge the battery at days with poor production. As the solar conditions  change yielding lower production, as shown in figure \ref{fig:prod_theoretic_chiwoza_20231207-20240115}, it is expected that the current control system will struggle to support demand for those days. The plot for the errors, shown in figure \ref{fig:error_231224-240101_CS_soiling26}, is as expected because the errors follow immediately after the days of low solar production. The current control system does not reduce the consumption, as shown by the satisfaction of flexible loads in \ref{fig:error_def_231224-240101_CS_soiling26}, which produces a low state of charge, leading to the system reaching battery minimum during the night before charging commences.

In contrast, the proposed control system has both production and consumption forecasts as inputs. This enables it to predict and respond to diminishing solar yield by lowering the supply to the flexible loads. This effect is shown in figure \ref{fig:error_def_231224-240101_tuning1_soiling26} and \ref{fig:error_def_231224-240101_tuning2_soiling26} for tuning 1 and 2 respectively by the higher negative deviation from flexible load demand for day 4 and 6. As these are the two days with lower production, reducing flexible loads allows the system to charge up to a high enough SOC to support critical loads until the next day. Compared to the current control system, the proposed control system is therefore considerably more effective in responding to changing conditions, fulfilling specification $4$ from table \ref{tab:control_system_specification}.\\

As the proposed control system has only been applied to one site, the specification $5$ of applicability across sites can not be fully answered for the specific derived solution itself. However, the process outlined, starting with building the forecaster and then the optimizer contains no steps unique to the case considered in this thesis. Repeating the process for another site of the same topology, i.e. a PV-powered islanded system, will therefore lead to a solution. The effort required to do so will depend on how similar the conditions are. However, considering the results from the user survey in appendix \ref{appendix:user_survey}, the similarity in terms of loads connected and the priority of the loads are large between the various sites, suggesting that much of the process can be replicated. One of the weaknesses of the current control system was the inability to adjust to the conditions at the site, noticeable in how some sites charged above the battery health threshold even when not required. The proposed solution would punish exceeding the threshold regardless of which site it is installed at.

Furthermore, as seen in figure \ref{fig:cr_reliability}, the proposed control system performed far better regarding critical load reliability than the current control system when the battery capacity was reduced. This indicates that the control system could perform well at sites with lower battery capacity. In all, there is a high likelihood of the proposed control system being more applicable than the current across different sites.\\  

The proposed control system has three weaknesses visible in the result: 1)the decrease in total reliability, 2) the oversupply to flexible load and 3) the decrease in utilization. The decrease in total reliability is visible as the increase of about 1 percentage point in total reliability in table \ref{tab:combined_results_load}. Considering the errors for the proposed control system, shown in figure \ref{fig:error_231224-240101_tuning1_soiling26} and figure \ref{fig:error_231224-240101_tuning2_soiling26} these errors are more frequent than the errors for the current control system, shown in figure \ref{fig:error_231224-240101_CS_soiling26}. Relating it to the power flow plots in figure \ref{fig:pf_231224-240101_tuning1_soiling26} and \ref{fig:pf_231224-240101_tuning2_soiling26}, the errors are occurring during times of battery discharge. This indicates two related weaknesses in the design of the proposed control system. The first of which is the relatively high error from the forecasting for both the production and forecast. As the optimizer during times of battery discharge allocates an amount of power from the battery to exactly support the demand at that time-step, any error from the forecaster will lead to the optimizer allocating the wrong amount of energy. Hence, the lowest priority loads will be shed if the demand exceeds the allocation. To shed low priority load is the control mechanism of the control system, and to be expected. The sub-optimal situation that does arise however is when the control system allocates too little energy from the battery to the load because of a faulty forecast and hence shedding it when it could have been supplied. Evidence for this happening is seen by the increase in total reliability of 2.3 percentage points from the proposed control system with tuning 2 with perfect forecast compared to the proposed control system with tuning 2 with imperfect forecast.\\

The oversupply of flexible loads points to another weakness of the proposed control system. This is evident in both figure \ref{fig:error_def_231224-240101_tuning1_soiling26} and \ref{fig:error_def_231224-240101_tuning2_soiling26} where flexible loads are frequently supplied far more than demand. While not in itself a problem, it is sub-optimal as energy over-allocated to one load could instead have been used to supply another load. These results were at first surprising, because the objective function in \ref{eq:objective_function} does not reward, but punishes supply exceeding demand for flexible loads. Some of the over-supply can be attributed to inaccurate forecasting. However, even with perfect demand forecasting, the results shown in \ref{fig:error_def_231224-240101_tuning2_perfectR} have a persistent oversupply. This suggests imperfect forecasting is not the only reason. The objective function is non-linear,  The non-linearity of the optimizer can fail to yield a global minima, but instead be stuck in a local one. This yields sub-optimal results.

The post-processing of the optimizer solution is also unfortunate, because it hides details away from the optimizer, creating a distance between its dynamics and the system dynamics. This includes the fact that, with the exception of flexible loads, a load's demand either has to be fully met or not. This is hidden for the optimizer, which delegates an amount of energy to each load between 0 and its forecasted demand $r$. An integer programming approach was attempted, where the allocation would be binary 0 or $r$, but this failed. These two problems, together with the errors in forecasting, yield a non-optimal solution.\\

The decrease in utilization is a weakness of the solution, as this was indeed one of the KPIs from \ref{tab:KPI}. The decrease is visible in table \ref{tab:combined_results_load} where both tunings of the proposed control system have a decrease in utilization of about 20 percentage points. Looking at the figures, comparing the plot of the utilized and potential production in figure \ref{fig:prod_231224-240101_CS_soiling26} to the same plot for tuning 1 of the proposed control system in figure \ref{fig:prod_231224-240101_tuning1_soiling26}, we see that the difference between the potential and utilized production is larger. This means that more of the potential production remains unutilized. A decreased utilization compared to current control system is however expected, as the current control system has a very high utilization because it never limits loads. The lower utilization of the proposed control system suggests an area of improvement where additional loads could be run.

Comparing the results to the literature, the improvement in some objectives in this thesis mirrors the improvement \textit{(Salazar A. et al., 2020)} achieved by their optimization approach compared to the rule-based approach. Their optimization methods are similar to the ones used in this thesis, however, their control objectives were far different. The results are similar to the ones achieved by \textit{(Rajbhandari et al., 2022)} for their case study on implementing a rule-based control system to increase user satisfaction in a rural microgrid. Their results showed an increase in the energy supplied to high-priority loads, while a decrease in the total supply. This is the same result as in this thesis. The user satisfaction function defined by the authors through interaction with the community nevertheless yields a higher user satisfaction score to the proposed control system than the present control system. Because no user satisfaction function was created for this thesis, a similar conclusion can not be drawn. However, this thesis is able to reproduce the same effect of higher achievement on pre-defined goals, also at a broader set of goals than considered in that paper. The results also reflect the proposition from \textit{(Mehra, V. et al., 2018)}, that a controlled system can achieve higher critical load reliability than an uncontrolled one, even when having less battery capacity available.\\

The significance of this study is that it provides a process to create an adaptable control system shown through simulation to be able to respond to a wide and varying set of goals. While it fails to provide an optimal solution, it makes DCP able to improve part of its operation at sites. It therefore allows DCP to engage in further dialogue with its stakeholders as to how to prioritize between certain goals. This dialogue is a crucial part of the process, given the hesitation some end-users expressed in \ref{seq:acceptability} to the prospect of having an automatic control system implemented at their sites.\\

The proposed control system allows DCP to run their systems to conserve battery health and prioritize important loads. From an economic standpoint, a gain in battery health could translate into less need for battery replacement, reducing costs. Furthermore, figure \ref{fig:cr_reliability} showed that the proposed control system managed to keep a high critical load reliability even with less installed battery capacity. This gives DCP the option to reduce their system size while still being confident that prioritized loads can be supported. Referring back to the work by \textit{(Mehra, V. et al., 2018)} on system sizing and control, this can be turned into a lower \textit{Levelized Cost of Energy}, which again lowers the capital requirement for installing microgrids and providing energy access. The study therefore has significance for the UN Sustainable Development Goal 7\cite{un_sdg_7}.

Furthermore, the study opens several lines of further work to improve the proposed solution. First of all the solution is limited by the accuracy of the forecasts. For the production, this could be improved by adding measurement devices at the site allowing the measurement of unattenuated production. Because the current production measurement is limited by the load, its value for the estimation of total potential production is reduced. A possible solution could be a small stand-alone PV-module, consisting of a single panel, connected to a load consuming all the produced power. This would give insight into the actual production conditions at the facility. 

The load forecast was shown to be inaccurate for several loads. Most of the load forecasts consisted of a pure statistical analysis. The loads with sudden and rare spikes in their consumption proved difficult to forecast. To augment the statistical analysis, a study of the sites and their load usage could yield insight. This would be akin, but more in-depth than what was done in the user survey presented in this thesis, which mostly focused on prioritization between loads. 

Furthermore, the optimizer exhibited several instances of non-optimal behaviour. Before implementing at sites, DCP should aim to further explore and develop to fill these shortcomings. This could be by attempting to define the optimization problem discretely, as a mixed integer programming problem. This would address the issue of the optimizer allocating less than demanded to certain loads. The optimizer would also benefit from being defined using a differentiable objective function because that would increase the likelihood of achieving the global optimum. 

Lastly, there are several more steps before such a control system can be realised in practice. This includes hardware design and selection, designing a good human-machine interaction and communicate with all stakeholders. This is a multi-disciplinary effort that would require several diverse viewpoints and skill sets. By pointing out some possible solutions, and possible gains, this thesis is an initial step of that effort.