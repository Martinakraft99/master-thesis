\section{Background}

In 2015, the \textit{United Nations} (UN) proposed and adopted the 17 \textit{Sustainable Development Goals} (SDG) in the 2030 Agenda for Sustainable Development\cite{2030agenda} as a succession to the former Millennium Development Goals. Of these, SDG number 7 is about the \textit{availability of affordable and clean energy for all}\cite{un_sdg_7}. While the developed world is looking to replace its current energy supply with cleaner sources, energy, much less clean energy, remains unavailable for large parts of the developing world. This problem is especially acute in Africa. According to the UN report on the SDGs from 2023, 675 million people live without access to electricity, most of which in located in Sub-Saharan Africa.\cite{Desa2023-mr} Many of these live  in rural communities where the regional electricity grid is either unreliable or unreachable. The problem of electrifying these communities, powering and extending the reach of vital services in both health and education, is known as \textit{last-mile-electrification}.\\

Localised, small-scale grid networks using renewable resources have been proposed and developed in several locations where a grid connection is techno-economically infeasible. This is especially relevant in Africa with its abundant solar resources, but lack of infrastructure.\cite{Wang2021-kb} These networks may range from the larger microgrids to the smaller nanogrids depending on the size of the service it is providing. Common for all the solutions is that they provide a self-sustaining system, meaning it can operate without being connected to the larger grid.\\

The company \textit{Differ Community Power} (DCP) is a Norwegian private company looking to develop the technologies and business models to deploy microgrids across rural communities mostly located in rural Sub-Saharan Africa. The company already have more than 100 installations in operation across 8 different countries. The installations are often mounted to support some social infrastructure like health stations, vaccine dispensaries or schools with electricity. The installations are largely funded by multilateral or bilateral organisations and foundations such as UN, WFP, USAID etc. with the recipient government paying a smaller part.\\

In charity and development, it is generally known that it is easier to receive grants to install systems than to operate systems. This is because operation requires larger overhead spending, something organisations and donors loathe. After all, low overhead spending has been perceived as a key indicator of an organisation's effectiveness.\cite{Berrett2020-wi} This is unfortunate for several reasons. First, the competence the company could gain from assessing and evaluating over a longer period is lost. This competence could be used to better operations to offer better products for both the end-user and the customer. Secondly, a longer contract offers the end-user reliability and consistency over a longer period. Lastly, longer contracts create a sense of ownership and facilitate mutual information sharing between user and supplier. This incentivizes the training of the end-user to be able to perform maintenance and repair of the installations. The downsides of the neglect of operations and training have been clear to DCP when installing new systems. Frequently they have arrived at sites with systems already installed, but these are either broken or incomplete.

DCP offers a different approach with a complete package that includes both the \textit{Engineering, Procurement and Construction} (EPC) of the system and a long-term \textit{Operation and Monitoring} (O\&M). Already in place at their sites are systems to monitor the installations. What is lacking, and is of keen interest for DCP to develop, is a framework to analyse and act on, i.e. control, the systems to achieve improved operation on key indicators. This thesis delves into all these points: the analytics of the systems, how they can be controlled for better operation, and what those key indicators should be. 

\section{Motivation}

There are several stake-holders connected to a site, such as end-users, customers and the company DCP itself. Each of these are concerned with different parts of the operation. In this thesis, a series of \textit{Key Performance Indicators} (KPIs) are developed to capture the different interests of the stakeholders. The motivation behind the control system is then to improve the operation with regard to the KPIs. This translates into higher system reliability, availability and lifetime. Something that is of keen interest to all stakeholders.\\

DCP wants to increase the amount of installations in its portfolio. The cost and time required to perform O\&M on the installations increase with the amount of sites. Hence it becomes important to automatize as much as possible. An automated control system aims to fulfil 4 main motivations.\\

\begin{enumerate}
    \item \textbf{Load prioritization}  -    Some loads are critical to fulfilling the purpose of the facility. For instance, medical equipment is critical to fulfil the purpose of a health facility. These are generally deemed as more important by all stakeholders. Currently, these loads are not prioritized, neither in their immediate nor future demand. This means that nothing is stopping less important load from occupying all inverter capacity, or draining the battery so that a critical load cannot run. This is damaging for the end-users because vital services can be unavailable and unreliable. It is damaging for the company because it cannot guarantee to deliver reliable operation of critical load. And it is damaging for the customer because they are not getting the value in terms of human development for their investment.
    \item \textbf{Extend lifetime}  -   As the current control system is not controlling the operation effectively, the electrical system is running in a way which produces unnecessary degradation to various components. Certain components, like batteries, are expensive and prone to damage by unhealthy usage. If one could extend the lifetime of the battery, by improving the operation with regard to its health, it could mean a cost reduction for the long-term O\&M-contract.
    \item \textbf{Enabling additional loads}    -   Because O\&M is costly, DCP is exploring the possibility of offsetting some of that cost by installing additional revenue-generating loads. Examples of these can be rental portable batteries, a solar maize mill or refrigerated storage. The goal of these loads is to provide more streams of income for DCP so that the price of O\&M can be lowered. A control system is identified as a key enabler of such loads because the loads cannot run at the cost of the loads supporting the primary purpose of the installation. A control system could in theory decide to run loads only when surplus power is available.
    \item \textbf{System sizing}    -    Be gaining more insight and the ability to influence system operation, the hope is that this could translate into a more tailored installation size. \textit{(Mehra, V. et al., 2018)} provides a function relating the cost of system unavailability to the cost of various energy sources.\cite{Mehra2018-xs} They find that a control system that can reduce the unavailability of critical loads can justify a lower installed battery and \textit{photovoltaic}(PV)-capacity, reducing the \textit{levelized cost of energy} (LCOE). Most sites are today oversized with regards to production, meaning they most days consume far less energy than the systems could theoretically produce. If one could reduce the system size while keeping reliability and availability above an acceptable level, the cost of the installation could be reduced for the customer. Alternatively, one could provide more services for the end-users, enhancing the value provided by the site.
\end{enumerate}

The process of increasing the amount of loads is already happening, both from the top-down, with the customer donating equipment to support the facility's purpose, or from bottom-up, by end-users such as staff buying electronic devices for their daily life. \textit{Increased access to energy creates a bigger energy demand}. The challenges listed above of system sizing, prioritization and lifetime are only going to become more and more pressing. This thesis arrives at an opportune moment.\\


Provided here is an attempt to design a solution satisfying the issues outlined in the motivation. Specifically this thesis will consider the site of Chiwoza, a small rural health facility located outside the Malawian capital of Lilongwe. The electrical system supporting the operation at the site is studied an analysed. The electrical consumption is analysed through a statistical study and by performing a user survey. The production is studied through a physical model of its sole energy source, a PV-module. The physical model relates the PV-module production with the solar irradiance expected at the site. Based on the analysis a statistical forecasting method is designed to forecast the consumption, while a physical model is developed to forecast production. These are combined with a non-linear optimizer to control the consumption and battery charging. The system is designed using a set of historical data, and then tested by simulation using another, disjoint set of historical data. The results from this simulation show a reduction in unavailability for the critical load, together with better operation for battery health. This although comes with the cost of a higher general unavailability and lower system utilization compared to the current control system. 

\section{Thesis structure}
 In the endeavour of creating a new control system, first, a survey of existing research relevant to the problem is conducted. The literature survey includes related works within \textit{Load analysis, Load Forecasting, Production Analysis, Production forecasting and Energy Management Systems}. The survey is found in \autoref{chap:related_works}. Some deductions and results are included in the proceeding \autoref{chap:theory} to not congest the following chapters. The chapter contains theoretical results on photovoltaic cells, forecasting and battery health used in the rest of the thesis. The subsequent chapters concern the case at hand specifically. In \autoref{chap:system_overview} the current system is outlined. The following \autoref{chap:design} proposes a solution design, with the first part giving the specifications for which the design is to be evaluated. This chapter also includes the necessary analysis for the synthesis of the solution. In \autoref{chap:implementation} the solution is implemented in a simulation environment, with \autoref{chap:results} showing the results of that simulation. These results, together with the design are then evaluated in \autoref{chap:discussion}. This chapter also contains suggestions for future works related to the subject in this thesis. The final \autoref{chap:conclusion} concludes the work done for this thesis.\\

 The main contributions of this thesis are:
\begin{itemize}
    \item \textbf{Propose a control system} yielding improved operations for identified KPIs, with the possibility to adjust the prioritization of the KPIs. 
    \item \textbf{A method for developing load forecasters} using historic data.
    \item \textbf{A proposed physical model for production forecasting}.
    \item \textbf{A system model and simulation} allowing DCP to predict and evaluate their operation.
\end{itemize}
